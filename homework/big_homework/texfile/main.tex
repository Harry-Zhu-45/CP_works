\documentclass[bibend=bibtex,lang=cn,11pt,a4paper]{elegantpaper}

\title{2022 计算物理期末结课论文}
\author{祝茗 \\ 武汉大学}
% \institute{\href{https://elegantlatex.org/}{Elegant\LaTeX{} 项目组}}
% \version{1.0}
\date{}


% 本文档命令
\usepackage{array}
\usepackage{graphicx}
\usepackage{pythonhighlight}
\newcommand{\ccr}[1]{\makecell{{\color{#1}\rule{1cm}{1cm}}}}


% 正文内容
\begin{document}

\maketitle

\begin{abstract}
  本文为 $2022$ 计算物理课程的结课论文。内容主要包含了相变模型与元胞自动机。
\end{abstract}

\section{背景介绍}

在学习和使用了一个学期的 Python 后,已经可以使用 matplolib 的库来进行基础绘图。

在学习计算物理的过程中,我学到了许多有趣的知识与模型。其中最让我感兴趣的是有关 2D Ising model 的模拟,因为在 Ising model 通过简单的模型与简单的模拟,可以得到并不简单的相变过程,这也可以算是一种 "more is difference" \cite{MoreIsDifferent}。

在本学期的学习过程中,采用了 Monte Carlo 方法来模拟 2D Ising model ,但是在思索期末课程论文选题的过程中,我浏览到了一个同样可以用来描述相变过程的模型 —— \href{https://www.youtube.com/watch?v=a-767WnbaCQ}{Percolation: a Mathematical Phase Transition},不同于 通过多次 Monte Carlo 方法实现的 2D Ising model, Percolation model 只需要一次随机的初始化就可以通过调节 percolation 的概率来获得一个单调递增的模拟结果,比 2D Ising Model 的实现更加的稳定与高效。在学习与实现 Percolation model 的过程中,我还留意到了另外一种模拟相变的模型——元胞自动机 (Cellular Automata),也进行了简单的学习与实现。

\section{Percolation 与 Cellular Automata}

通过编写一个 \lstinline{PercolationModel2D} 的类,来实现模型中网格的存储,详细的代码可以参见附录。由前人的探索,可知在 $p=0.45$ 附近会发生相变,因此在一定范围内调整 $p$ 的大小可以观察到相变的发生。通过先输出多张随时间演化的图片,后手动筛选演化终止的结点,选择合适的截止位置,然后将多张 png 格式的图片拼接为 gif 作为结果的呈现。

\section{结论}

在 $p=0.45$ 附近会发生相变,percolation model 几乎可以占据整个空间。

\section{附录}

\subsection{Class PercolationModel2D}

\begin{python}
import numpy as np
class PercolationModel2D(object):
    """
    Object that calculates and displays behaviour of 2D cellular automata
    """

    def __init__(self, ni):
        """
        Constructe a 2D cellular automaton
        input:
            ni: number of cells in each direction
        """
        self.N = ni                # Number of cells in each direction
        self.Ntot = self.N*self.N  # Total number of cells

        self.grid = np.zeros((self.N, self.N))
        self.nextgrid = np.zeros((self.N, self.N))
        self.tested = np.zeros((self.N, self.N))

        self.complete = False  # Boolean to indicate whether the model has completed, default is False

    def getMooreNeighbourhood(self, i, j):
        """
        Return a set of indices corresponding to the Moore Neighbourhood, i.e. the cells immediately adjacent to (i,j) and the cells diagonally adjacent to (i,j)
        input:
            `i`: row index
            `j`: column index
        output:
            `indices`: list of indices corresponding to the Moore Neighbourhood
        """
        indices = []
        for iadd in range(i-1, i+2):
            for jadd in range(j-1, j+2):
                if (iadd == i and jadd == j):
                    continue  # exclude the cell itself

                if (iadd > self.N-1):
                    iadd = iadd - self.N  # periodic boundary conditions
                if (jadd > self.N-1):
                    jadd = jadd - self.N  # periodic boundary conditions

                indices.append([iadd, jadd])
        return indices

    def getVonNeumannNeighbourhood(self, i, j):
        """
        Return a set of indices corresponding to the Von Neumann Neighbourhood, i.e. the cells immediately adjacent to (i,j)
        input:
            `i`: row index
            `j`: column index
        output:
            `indices`: list of indices corresponding to the Von Neumann Neighbourhood
        """
        indices = []
        for iadd in range(i-1, i+2):
            if (iadd == i):
                continue
            if (iadd > self.N-1):
                iadd = iadd - self.N
            indices.append([iadd, j])

        for jadd in range(j-1, j+2):
            if (jadd == j):
                continue
            if (jadd > self.N-1):
                jadd = jadd - self.N
            indices.append([i, jadd])
        return indices

    def check_complete(self):
        """
        Check if all cells have been tested
        output:
            `complete`: boolean, whether the model has completed
        """
        ntested = np.sum(self.tested)   # number of cells that have been tested
        if (ntested == self.N*self.N):  # if all cells have been tested
            self.complete = True        # indicate the model has completed
        return self.complete

    def randomise(self):
        """
        Place a random selection of zeros and ones into grid
        """
        for i in range(self.N):
            for j in range(self.N):
                self.grid[i, j] = np.rint(np.random.random())

    def randomise_with_symmetry(self):
        """
        Place a random selection of zeros and ones into grid, with centual symmetry
        """
        for i in range(self.N/2):
            for j in range(self.N/2):
                self.grid[i, j] = np.rint(np.random.random())
                self.grid[i+self.N/2, j] = self.grid[i, j+self.N/2] = self.grid[i+self.N/2, j+self.N/2] = self.grid[i, j]

    def clear(self, icentre, jcentre, extent):
        """
        Clear a space on the grid
        input:
            `icentre`: row index of centre of space to clear
            `jcentre`: column index of centre of space to clear
            `extent`: extent of space to clear
        """
        for i in range(icentre-extent, icentre+extent):
            for j in range(jcentre-extent, jcentre+extent):
                if (i > 0 and i < self.N and j > 0 and j < self.N):
                    self.grid[i, j] = 0

    def updateGrid(self):
        """
        Take the changes queued up on self.nextgrid, and applies them to self.grid
        """
        self.grid = np.copy(self.nextgrid)
        self.nextgrid = np.zeros((self.N, self.N))

    def ApplyPercolationModelRule(self, P):
        """
        Construct the self.nextgrid matrix based on the properties of self.grid
        Applie the Percolation Model Rules:
            1. Cells attempt to colonise their Moore Neighbourhood with probability P
            2. Cells do not make the attempt with probability 1-P
        """
        for i in range(self.N):
            for j in range(self.N):
                # If cell has already been tested
                if (self.tested[i, j] == 1):
                    self.nextgrid[i, j] = self.grid[i, j]  # Copy value from self.grid to self.nextgrid
                    continue                               # Skip to next cell

                # If cell contains a coloniser, then decide whether to colonise
                if (self.grid[i, j] == 1 and self.tested[i, j] == 0):
                    # If colonisation occurs
                    if (np.random.rand() < P):
                        self.nextgrid[i, j] = 1
                        indices = self.getMooreNeighbourhood(i, j)

                        for element in indices:
                            if (self.tested[element[0], element[1]] == 1):
                                continue
                            if (self.grid[element[0], element[1]] == 0):
                                self.nextgrid[element[0], element[1]] = 1

                    else:
                        self.nextgrid[i, j] = -1  # If colonisation does not occur, then mark cell as tested

                    self.tested[i, j] = 1
\end{python}

\subsection{Cellular Automata Patterns}

\begin{python}
"""
This file contains functions to add various patterns to the percolation model
"""
from PercolationModel import PercolationModel2D


def add_block(cell: PercolationModel2D, icentre: int, jcentre: int):
    """
    Add a 2x2 block into the system, with bottom left corner (icentre, jcentre)
    like this:
        xx\n
        xx\n
    """
    extent = 2
    cell.clear(icentre, jcentre, extent)                 # clear the space
    cell.grid[icentre:icentre+2, jcentre:jcentre+2] = 1  # add the block


def add_beehive(cell: PercolationModel2D, icentre: int, jcentre: int):
    """
    Add a beehive into the system, with (icentre, jcentre) being the inner left blank square
    like this:
        oxxo\n
        xoox\n
        oxxo\n
    """
    extent = 7
    cell.clear(icentre, jcentre, extent)
    cell.grid[icentre-1, jcentre:jcentre+2] = 1  # top row
    cell.grid[icentre+1, jcentre:jcentre+2] = 1  # bottom row
    cell.grid[icentre, jcentre-1] = 1            # left dot
    cell.grid[icentre, jcentre+2] = 1            # right dot


def add_blinker(cell: PercolationModel2D, icentre: int, jcentre: int):
    """
    Add a horizontal line of 3 blocks, a period 2 oscillator
    like this:
        xxx\n
    """
    extent = 4
    cell.clear(icentre, jcentre, extent)
    cell.grid[icentre, jcentre-1:jcentre+2] = 1


def add_loaf(cell: PercolationModel2D, icentre: int, jcentre: int):
    """
    Add a loaf, with (icentre, jcentre) in the bottom left corner (blank)   
    like this:
        ooxo\n
        oxox\n
        xoox\n
        oxxo\n
    """
    cell.grid[icentre, jcentre+1:jcentre+3] = 1
    cell.grid[icentre+1:icentre+3, jcentre+3] = 1
    cell.grid[icentre+1, jcentre] = 1
    cell.grid[icentre+2, jcentre+1] = 1
    cell.grid[icentre+3, jcentre+2] = 1


def add_boat(cell: PercolationModel2D, icentre: int, jcentre: int):
    """
    Add a boat, with (icentre,jcentre) in the bottom left corner (blank)
    like this:
        xxo\n
        xox\n
        oxo\n
    """
    extent = 4
    cell.clear(icentre, jcentre, extent)

    indices = cell.getVonNeumannNeighbourhood(icentre, jcentre)
    for element in indices:
        cell.grid[element[0], element[1]] = 1

    cell.grid[icentre+1, jcentre-1] = 1


def add_toad(cell: PercolationModel2D, icentre: int, jcentre: int):
    """
    Add a toad, a period 2 oscillator
    like this:
        xo\n
        xx\n
        xx\n
        ox\n
    """
    extent = 3
    cell.clear(icentre, jcentre, extent)
    cell.grid[icentre-1:icentre+2, jcentre] = 1
    cell.grid[icentre:icentre+3, jcentre-1] = 1


def add_beacon(cell: PercolationModel2D, icentre: int, jcentre: int):
    """
    Add two 2x2 blocks, which repeat a pattern of period 2
    like this:
        xxoo\n
        xxoo\n
        ooxx\n
        ooxx\n
    """
    extent = 3
    cell.clear(icentre, jcentre, extent)
    add_block(cell, icentre+2, jcentre)
    add_block(cell, icentre, jcentre+2)


def add_pulsar(cell: PercolationModel2D, icentre: int, jcentre: int):
    """
    Add a pulsar, a period 3 oscillator
    """
    extent = 8
    cell.clear(icentre, jcentre, extent)

    # Start with inner cross
    # North
    cell.grid[icentre+2:icentre+5, jcentre+1] = 1
    cell.grid[icentre+2:icentre+5, jcentre-1] = 1

    # South
    cell.grid[icentre-4:icentre-1, jcentre+1] = 1
    cell.grid[icentre-4:icentre-1, jcentre-1] = 1

    # East
    cell.grid[icentre+1, jcentre+2:jcentre+5] = 1
    cell.grid[icentre-1, jcentre+2:jcentre+5] = 1

    # West
    cell.grid[icentre+1, jcentre-4:jcentre-1] = 1
    cell.grid[icentre-1, jcentre-4:jcentre-1] = 1

    # Now do surrounding bars - quadrant at a time
    cell.grid[icentre+6, jcentre+2:jcentre+5] = 1
    cell.grid[icentre+2:icentre+5, jcentre+6] = 1

    cell.grid[icentre-4:icentre-1, jcentre+6] = 1
    cell.grid[icentre-6, jcentre+2:jcentre+5] = 1

    cell.grid[icentre-6, jcentre-4:jcentre-1] = 1
    cell.grid[icentre-4:icentre-1, jcentre-6] = 1

    cell.grid[icentre+2:icentre+5, jcentre-6] = 1
    cell.grid[icentre+6, jcentre-4:jcentre-1] = 1


def add_glider(cell: PercolationModel2D, icentre: int, jcentre: int):
    """
    Add a glider, with (icentre,jcentre) being the bottom left tile (alive) - period 4 oscillator
    like this:
        oxo\n
        oox\n
        xxx\n
    """
    cell.grid[icentre, jcentre:jcentre+3] = 1
    cell.grid[icentre+1, jcentre+2] = 1
    cell.grid[icentre+2, jcentre+1] = 1


def add_spaceship(cell: PercolationModel2D, icentre: int, jcentre: int):
    """
    Add a lightweight spaceship, with (icentre,jcentre) being the bottom left tile (alive) - period 4 oscillator
    like this:
        oxoox\n
        xoooo\n
        xooox\n
        xxxxo\n
    """
    cell.grid[icentre, jcentre:jcentre+4] = 1
    cell.grid[icentre:icentre+3, jcentre] = 1
    cell.grid[icentre+3, jcentre+1] = 1
    cell.grid[icentre+3, jcentre+4] = 1
    cell.grid[icentre+1, jcentre+4] = 1


def add_glider_gun(cell: PercolationModel2D, icentre: int, jcentre: int):
    """
    Add a Gosper glider gun (period 30 oscillator) with (icentre,jcentre) being the bottom left tile (alive)
    """
    add_block(cell, icentre+5, jcentre+2)

    # Make first of inner patterns
    cell.grid[icentre+4:icentre+7, jcentre+12] = 1
    cell.grid[icentre+3, jcentre+13] = cell.grid[icentre+7, jcentre+13] = 1

    # cell.grid[icentre+2,jcentre+14] = cell.grid[icentre+8,jcentre+14] = 1
    cell.grid[icentre+2, jcentre+14:jcentre+16] = cell.grid[icentre+8, jcentre+14:jcentre+16] = 1
    cell.grid[icentre+5, jcentre+16] = 1
    cell.grid[icentre+3, jcentre+17] = cell.grid[icentre+7, jcentre+17] = 1
    cell.grid[icentre+4:icentre+7, jcentre+18] = 1
    cell.grid[icentre+5, jcentre+19] = 1

    # Now second pattern
    cell.grid[icentre+6:icentre+9, jcentre+22:jcentre+24] = 1
    cell.grid[icentre+5, jcentre+24] = cell.grid[icentre+9, jcentre+24] = 1
    cell.grid[icentre+4:icentre+6, jcentre+26] = cell.grid[icentre+9:icentre+11, jcentre+26] = 1

    add_block(cell, icentre+7, jcentre+36)
\end{python}

\nocite{*}
\printbibliography[heading=bibintoc, title=\ebibname]

\appendix
% \appendixpage
\addappheadtotoc

\end{document}
